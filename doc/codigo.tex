\section{Código fuente}

Se omite el launcher del lexer dado que solamente es llamar al lexer desde un main y no forma parte del programa principal.

\subsection{tokrules.py}

\definecolor{codegreen}{rgb}{0,0.6,0}
\definecolor{codegray}{rgb}{0.5,0.5,0.5}
\definecolor{codepurple}{rgb}{0.58,0,0.82}
\definecolor{backcolour}{rgb}{0.95,0.95,0.92}

\definecolor{codegreen}{rgb}{0,0.6,0}
\definecolor{codegray}{rgb}{0.5,0.5,0.5}
\definecolor{codepurple}{rgb}{0.58,0,0.82}
\definecolor{backcolour}{rgb}{0.95,0.95,0.92}

\lstdefinestyle{mystyle}{
    backgroundcolor=\color{backcolour},
    commentstyle=\color{codegreen},
    keywordstyle=\color{magenta},
    numberstyle=\tiny\color{codegray},
    stringstyle=\color{codepurple},
    basicstyle=\footnotesize,
    breakatwhitespace=false,
    breaklines=true,
    captionpos=b,
    keepspaces=true,
    numbers=left,
    numbersep=5pt,
    showspaces=false,
    showstringspaces=false,
    showtabs=false,
    tabsize=2
}

\lstset{style=mystyle}
    \begin{lstlisting}[language=Python]
        import ply.lex as lex

        tokens = (
            'CHR',
            'DIVIDE'
        )

        # los literales son chars que matchean de una
        literals = "_^(){}"

        def t_CHR(t):
            # el primer ^ toma complemento de los simbolos que siguen
            r'[^ _ \^ / \( \)\{\}]'
            return t

        def t_DIVIDE(t):
            r'/'
            return t

        t_ignore  = '\t'

        def t_error(t):
            print("Illegal character '%s'" % t.value[0])
            t.lexer.skip(1)

    \end{lstlisting}

\subsection{AST.py}

    \begin{lstlisting}[language=Python]
        class Expr: pass

        class LambdaExpr(Expr):
            def __init__(self):
                pass

            def __eq__(self, other):
                return isinstance(other, LambdaExpr)

            def __str__(self):
                return "LambdaExpr"

            def accept(self, visitor):
                visitor.visitLambda(self)

        # a
        class Chr(Expr):

            non_character_error_description = "Character should have length one"

            def __init__(self, character):
                if len(character) != 1:
                    raise ValueError(Chr.non_character_error_description)
                self.character = character

            def __eq__(self, other):
                if isinstance(other, Chr):
                    return other.character == self.character
                else:
                    return False

            def __str__(self):
                return "Chr({0})".format(self.character)

            def accept(self, visitor):
                visitor.visitChr(self)

        class DivExpr(Expr):
            def __init__(self, leftExpr, rightExpr):
                self.leftExpr = leftExpr
                self.rightExpr = rightExpr

            def __eq__(self, other):
                if not isinstance(other, DivExpr):
                    return False

                return self.leftExpr == other.leftExpr and self.rightExpr == other.rightExpr
            def __str__(self):
                return "DivExpr({0},{1})".format(self.leftExpr, self.rightExpr)

            def accept(self, visitor):
                visitor.visitDiv(self)

        # A B
        class Concat(Expr):
            def __init__(self, leftExpression, rightExpression):
                self.leftExpression = leftExpression
                self.rightExpression = rightExpression

            def __eq__(self, other):
                if not isinstance(other, Concat):
                    return False
                return self.leftExpression == other.leftExpression and self.rightExpression == other.rightExpression

            def __str__(self):
                return "Concat({0},{1})".format(self.leftExpression, self.rightExpression)

            def accept(self, visitor):
                visitor.visitConcat(self)

        class SuperSub(Expr):
            def __init__(self, mainExpr, superExpr, subExpr):
                self.mainExpr = mainExpr
                self.superExpr = superExpr
                self.subExpr = subExpr

            def __eq__(self, other):
                if not isinstance(other, SuperSub):
                    return False

                comp = True
                comp = comp and (self.mainExpr == other.mainExpr)
                comp = comp and (self.superExpr == other.superExpr)
                comp = comp and (self.subExpr == other.subExpr)

                return comp

            def __str__(self):
                return "SuperSub({0}, {1}, {2})".format(str(self.mainExpr), str(self.superExpr), str(self.subExpr))

            def accept(self, visitor):
                visitor.visitSuperSub(self)

        class SubSuper(Expr):
            def __init__(self, mainExpr, subExpr, superExpr):
                self.mainExpr = mainExpr
                self.superExpr = superExpr
                self.subExpr = subExpr

            def __eq__(self, other):
                if not isinstance(other, SubSuper):
                    return False

                comp = True
                comp = comp and (self.mainExpr == other.mainExpr)
                comp = comp and (self.superExpr == other.superExpr)
                comp = comp and (self.subExpr == other.subExpr)

                return comp

            def __str__(self):
                return "SubSuper({0}, {1}, {2})".format(str(self.mainExpr), str(self.superExpr), str(self.subExpr))

            def accept(self, visitor):
                visitor.visitSubSuper(self)

        class SuperSuffix(Expr):
            def __init__(self, expr):
                self.expr = expr

            def __eq__(self, other):
                if not isinstance(other, SuperSuffix):
                    return False
                return other.expr == self.expr

            def __str__(self):
                return "SuperSuffix({0})".format(self.expr)

            def accept(self, visitor):
                visitor.visitSuperSuffix(self)

        class SubSuffix(Expr):
            def __init__(self, expr):
                self.expr = expr

            def __eq__(self, other):
                if not isinstance(other, SubSuffix):
                    return False
                return other.expr == self.expr

            def __str__(self):
                return "SubSuffix({0})".format(self.expr)

            def accept(self, visitor):
                visitor.visitSubSuffix(self)

        # ()
        class GroupedPar(Expr):
            def __init__(self, expr):
                self.expr = expr

            def __eq__(self, other):
                if not isinstance(other, GroupedPar):
                    return False
                return self.expr == other.expr

            def __str__(self):
                return "GroupedPar({0})".format(self.expr)

            def accept(self, visitor):
                visitor.visitGroupedPar(self)

    \end{lstlisting}

\subsection{parser\_rules.py}

    \begin{lstlisting}[language=Python]
        import ply.yacc as yacc
        from tokrules import tokens
        from AST import *

        SYNTAX_ERROR_IN_INPUT_ERROR_MESSAGE = "Syntax error in input!"

        precedence = (
            ('left', 'DIV'),

            ('left', '{', '('),

            ('left', 'CHR'),

            ('left', 'CONCAT'),

            ('nonassoc', '^'),
                ('nonassoc', '_'),

        )

        def p_unary(p):
            '''expression : unary_exp'''
            p[0] = p[1]

        def p_expression_chr(p):
            '''unary_exp : CHR'''
            p[0] = Chr(p[1])

        def p_expression_concat(p):
            '''expression : expression expression %prec CONCAT'''
            # %prec asocia la precedencia de la produccion a la del pseudosimbolo CONCAT
            p[0] = Concat(p[1], p[2])

        def p_expression_div(p):
            '''expression : expression DIVIDE expression %prec DIV'''
            p[0] = DivExpr(p[1], p[3])

        def p_expression_super(p):
            '''expression : unary_exp '^' unary_exp subexp'''
            p[0] = SuperSub(p[1], p[3], p[4])

        def p_expression_sub(p):
            '''expression : unary_exp '_' unary_exp superexp'''
            p[0] = SubSuper(p[1], p[3], p[4])

        def p_superexp_lambda(p):
            '''superexp : lambda'''
            p[0] = LambdaExpr()

        def p_superexp_expr(p):
            '''superexp : '^' unary_exp'''
            p[0] = SuperSuffix(p[2])

        def p_subexp_lambda(p):
            '''subexp : lambda'''
            p[0] = LambdaExpr()

        def p_subexp_expr(p):
            '''subexp : '_' unary_exp'''
            p[0] = SubSuffix(p[2])

        def p_expression_grouped_par(p):
            '''unary_exp : '(' expression ')' '''
            p[0] = GroupedPar(p[2])

        def p_expression_grouped_brkt(p):
            '''unary_exp : '{' expression '}' '''
            # Curly Brackets no aparecen en el AST
            p[0] = p[2]

        def p_lambda(p):
            '''lambda :'''

        def p_error(p):
            raise ValueError(SYNTAX_ERROR_IN_INPUT_ERROR_MESSAGE)

    \end{lstlisting}


\subsection{AST\_visitors.py}

    \begin{lstlisting}[language=Python]
        from AST import *
        from copy import copy

        class Visitor: pass

        class EscaleVisitor(Visitor):

            def __init__(self, escale):
                self.e = escale

            def visitLambda(self, expr):
                pass

            def visitChr(self, expr):
                expr.e = self.e

            def visitDiv(self, expr):
                expr.e = self.e
                expr.leftExpr.accept(self)
                expr.rightExpr.accept(self)

            def visitConcat(self, expr):
                expr.e = self.e
                expr.leftExpression.accept(self)
                expr.rightExpression.accept(self)

            def visitSubSuper(self, expr):
                expr.e = self.e
                expr.mainExpr.accept(self)
                expr.superExpr.accept(EscaleVisitor(0.7*self.e))
                expr.subExpr.accept(EscaleVisitor(0.7*self.e))

            def visitSuperSub(self, expr):
                expr.e = self.e
                expr.mainExpr.accept(self)
                expr.superExpr.accept(EscaleVisitor(0.7*self.e))
                expr.subExpr.accept(EscaleVisitor(0.7*self.e))

            def visitSuperSuffix(self, supersuffix_expr):
                supersuffix_expr.e = self.e
                supersuffix_expr.expr.accept(self)

            def visitSubSuffix(self, subsuffix_expr):
                subsuffix_expr.e = self.e
                subsuffix_expr.expr.accept(self)

            def visitGroupedPar(self, grouped_expr):
                grouped_expr.e = self.e
                grouped_expr.expr.accept(self)

        class WidthVisitor(Visitor):

            def __init__(self): pass

            def visitLambda(self, expr):
                expr.a = 0

            def visitChr(self, expr):
                expr.a = 0.6 * expr.e

            def visitDiv(self, expr):
                expr.leftExpr.accept(self)
                expr.rightExpr.accept(self)
                expr.a = max(expr.leftExpr.a, expr.rightExpr.a)

            def visitConcat(self, expr):
                expr.leftExpression.accept(self)
                expr.rightExpression.accept(self)
                expr.a = expr.leftExpression.a + expr.rightExpression.a

            def visitSubSuper(self, expr):
                expr.mainExpr.accept(self)
                expr.superExpr.accept(self)
                expr.subExpr.accept(self)
                expr.a = expr.mainExpr.a + max(expr.superExpr.a, expr.subExpr.a)

            def visitSuperSub(self, expr):
                expr.mainExpr.accept(self)
                expr.superExpr.accept(self)
                expr.subExpr.accept(self)
                expr.a = expr.mainExpr.a + max(expr.superExpr.a, expr.subExpr.a)

            def visitSuperSuffix(self, supersuffix_expr):
                supersuffix_expr.expr.accept(self)
                supersuffix_expr.a = supersuffix_expr.expr.a

            def visitSubSuffix(self, subsuffix_expr):
                subsuffix_expr.expr.accept(self)
                subsuffix_expr.a = subsuffix_expr.expr.a

            def visitGroupedPar(self, grouped_expr):
                grouped_expr.expr.accept(self)
                grouped_expr.a = grouped_expr.expr.a + grouped_expr.e * 2 * 0.6 # contar ()

        class HVisitor(Visitor):

            def __init__(self): pass

            def visitLambda(self, expr):
                expr.h1 = 0
                expr.h2 = 0
                expr.h = 0

            def visitChr(self, expr):
                expr.h1 = expr.e
                expr.h2 = 0
                expr.h = expr.e

            def visitDiv(self, expr):
                expr.leftExpr.accept(self)
                expr.rightExpr.accept(self)
                expr.h1 = expr.leftExpr.h1 + expr.leftExpr.h2 + expr.e*0.6
                expr.h2 = expr.rightExpr.h1 + expr.rightExpr.h2 - expr.e*0.6
                expr.h = expr.h1 + expr.h2

            def visitConcat(self, expr):
                expr.leftExpression.accept(self)
                expr.rightExpression.accept(self)
                expr.h1 = max(expr.leftExpression.h1, expr.rightExpression.h1)
                expr.h2 = max(expr.leftExpression.h2, expr.rightExpression.h2)
                expr.h = expr.h1 + expr.h2

            def visitSubSuper(self, expr):
                expr.mainExpr.accept(self)
                expr.superExpr.accept(self)
                expr.subExpr.accept(self)

                super_h1 = 0 if isinstance(expr.superExpr, LambdaExpr) else expr.mainExpr.h1*0.45 + expr.mainExpr.e + expr.superExpr.h1 - expr.superExpr.e
                #sumamos el h1 hasta el 'y' del superindice con su h1 y le restamos su escala (que sino se suma dos veces)

                expr.h1 = max(expr.mainExpr.h1, super_h1)

                sub_h2 = expr.subExpr.h2 + expr.subExpr.e + expr.mainExpr.h*0.25 - expr.mainExpr.e*0.7 # en un dibujo se ve bien, notar el 0.7 porque los char del mainExpr no miden toda la escala de largo sino que solo el 70% y el restante es espacio vacio
                expr.h2 = max(expr.mainExpr.h2, sub_h2)
                expr.h = expr.h1 + expr.h2

            def visitSuperSub(self, expr):
                expr.mainExpr.accept(self)
                expr.superExpr.accept(self)
                expr.subExpr.accept(self)

                sub_h2 = 0 if isinstance(expr.subExpr, LambdaExpr) else expr.subExpr.h2 + expr.subExpr.e + expr.mainExpr.h*0.25 - expr.mainExpr.e*0.7

                expr.h2 = max(expr.mainExpr.h2, sub_h2)
                super_h1 = expr.mainExpr.h1*0.45 + expr.mainExpr.e + expr.superExpr.h1 - expr.superExpr.e
                expr.h1 = max(expr.mainExpr.h1, super_h1)
                expr.h = expr.h1 + expr.h2

            def visitSuperSuffix(self, supersuffix_expr):
                supersuffix_expr.expr.accept(self)
                supersuffix_expr.h1 = supersuffix_expr.expr.h1
                supersuffix_expr.h2 = supersuffix_expr.expr.h2
                supersuffix_expr.h = supersuffix_expr.h1 + supersuffix_expr.h2

            def visitSubSuffix(self, subsuffix_expr):
                subsuffix_expr.expr.accept(self)
                subsuffix_expr.h1 = subsuffix_expr.expr.h1
                subsuffix_expr.h2 = subsuffix_expr.expr.h2
                subsuffix_expr.h = subsuffix_expr.h1 + subsuffix_expr.h2

            def visitGroupedPar(self, grouped_expr):
                grouped_expr.expr.accept(self)
                grouped_expr.h1 = grouped_expr.expr.h1
                grouped_expr.h2 = grouped_expr.expr.h2
                grouped_expr.h = grouped_expr.expr.h

        class XVisitor(Visitor):

            def __init__(self, pos):
                self.pos = pos

            def visitLambda(self, expr): pass

            def visitChr(self, expr):
                expr.x = self.pos

            def visitDiv(self, expr):
                divwidth = max(expr.leftExpr.a, expr.rightExpr.a)
                #ambos centrados respecto de la linea de division
                expr.leftExpr.accept(XVisitor(self.pos + (divwidth - expr.leftExpr.a)/2))
                expr.rightExpr.accept(XVisitor(self.pos + (divwidth - expr.rightExpr.a)/2))
                expr.x = self.pos

            def visitConcat(self, expr):
                expr.leftExpression.accept(self)
                expr.rightExpression.accept(XVisitor(self.pos + expr.leftExpression.a))
                expr.x = self.pos

            def visitSubSuper(self, expr):
                expr.mainExpr.accept(self)
                expr.subExpr.accept(XVisitor(self.pos + expr.mainExpr.a))
                expr.superExpr.accept(XVisitor(self.pos + expr.mainExpr.a))
                expr.x = self.pos

            def visitSuperSub(self, expr):
                expr.mainExpr.accept(self)
                expr.superExpr.accept(XVisitor(self.pos + expr.mainExpr.a))
                expr.subExpr.accept(XVisitor(self.pos + expr.mainExpr.a))
                expr.x = self.pos

            def visitSuperSuffix(self, supersuffix_expr):
                supersuffix_expr.expr.accept(self)
                supersuffix_expr.x = self.pos

            def visitSubSuffix(self, subsuffix_expr):
                subsuffix_expr.expr.accept(self)
                subsuffix_expr.x = self.pos

            def visitGroupedPar(self, grouped_expr):
                grouped_expr.expr.accept(XVisitor(self.pos + 0.6*grouped_expr.e)) # dejar espacio para '('
                grouped_expr.x = self.pos

        class YVisitor(Visitor):

            def __init__(self, pos):
                self.pos = pos

            def visitLambda(self, expr): pass

            def visitChr(self, expr):
                expr.y = self.pos

            def visitDiv(self, expr):
                expr.leftExpr.accept(YVisitor(self.pos - expr.leftExpr.h2 - expr.e*0.6))
                expr.rightExpr.accept(YVisitor(self.pos + expr.rightExpr.h1 - expr.e*0.6))
                expr.y = self.pos

            def visitConcat(self, expr):
                expr.leftExpression.accept(self)
                expr.rightExpression.accept(self)
                expr.y = self.pos

            def visitSubSuper(self, expr):
                expr.mainExpr.accept(self)
                expr.subExpr.accept(YVisitor(self.pos + expr.mainExpr.h*0.25)) #parece ser la formula que usaron en el ejemplo del enunciado
                expr.superExpr.accept(YVisitor(self.pos - expr.mainExpr.h1*0.45)) #visto en clase
                expr.y = self.pos

            def visitSuperSub(self, expr):
                expr.mainExpr.accept(self)
                expr.subExpr.accept(YVisitor(self.pos + expr.mainExpr.h*0.25))
                expr.superExpr.accept(YVisitor(self.pos - expr.mainExpr.h1*0.45))
                expr.y = self.pos

            def visitSuperSuffix(self, supersuffix_expr):
                supersuffix_expr.expr.accept(self)
                supersuffix_expr.y = self.pos

            def visitSubSuffix(self, subsuffix_expr):
                subsuffix_expr.expr.accept(self)
                subsuffix_expr.y = self.pos

            def visitGroupedPar(self, grouped_expr):
                if isinstance(grouped_expr.expr, DivExpr):
                    grouped_expr.expr.accept(YVisitor(self.pos - 0.3 * grouped_expr.expr.e))
                else:
                    grouped_expr.expr.accept(self)

                grouped_expr.y = self.pos

        class SVGRendererVisitor(Visitor) :

            def __init__(self): pass

            def visitLambda(self, expr):
                expr.svg = ""

            def visitChr(self, expr):
                expr.svg = "<text x=\"{}\" y=\"{}\" font-size=\"{}\">{}</text> \n".format(expr.x, expr.y, expr.e, expr.character)

            def visitDiv(self, expr):
                expr.leftExpr.accept(self)
                expr.rightExpr.accept(self)
                expr.svg = expr.leftExpr.svg
                expr.svg += "<line x1=\"{}\" y1=\"{}\" x2=\"{}\" y2=\"{}\" stroke-width=\"0.03\" stroke=\"black\"/>".format(expr.x, expr.y-expr.e*0.45, expr.x + expr.a, expr.y-expr.e*0.45)
                expr.svg += expr.rightExpr.svg

            def visitConcat(self, expr):
                expr.leftExpression.accept(self)
                expr.rightExpression.accept(self)
                expr.svg = expr.leftExpression.svg + expr.rightExpression.svg

            def visitSubSuper(self, expr):
                expr.mainExpr.accept(self)
                expr.superExpr.accept(self)
                expr.subExpr.accept(self)
                expr.svg = expr.mainExpr.svg + expr.subExpr.svg + expr.superExpr.svg

            def visitSuperSub(self, expr):
                expr.mainExpr.accept(self)
                expr.superExpr.accept(self)
                expr.subExpr.accept(self)
                expr.svg = expr.mainExpr.svg + expr.superExpr.svg + expr.subExpr.svg

            def visitSuperSuffix(self, supersuffix_expr):
                supersuffix_expr.expr.accept(self)
                supersuffix_expr.svg = supersuffix_expr.expr.svg

            def visitSubSuffix(self, subsuffix_expr):
                subsuffix_expr.expr.accept(self)
                subsuffix_expr.svg = subsuffix_expr.expr.svg

            def visitGroupedPar(self, grouped_expr):
                grouped_expr.expr.accept(self)
                times_scale = grouped_expr.h/grouped_expr.e
                grouped_expr.svg = "<text x=\"0\" y=\"0\" font-size=\"{}\" transform=\"translate({},{}) scale(1,{})\">(</text> \n".format(grouped_expr.e, grouped_expr.x, grouped_expr.y-(times_scale-1)*0.2*grouped_expr.e, times_scale)

                grouped_expr.svg += grouped_expr.expr.svg

                grouped_expr.svg += "<text x=\"0\" y=\"0\" font-size=\"{}\" transform=\"translate({},{}) scale(1,{})\">)</text> \n".format(grouped_expr.e, grouped_expr.x + grouped_expr.a - 0.6*grouped_expr.e, grouped_expr.y-(times_scale-1)*0.2*grouped_expr.e, times_scale)

    \end{lstlisting}


\subsection{ParserTest.py}

    \begin{lstlisting}[language=Python]
        import unittest
        import main
        from AST import *
        from AST_visitors import *

        class ParserTest(unittest.TestCase):
            def test_cant_create_chr_expression_with_non_character(self):
                with self.assertRaises(ValueError) as contextManager:
                    chr = Chr("")
                self.assertEqual(str(contextManager.exception), Chr.non_character_error_description)

            def test_character_is_chr_expression(self):
                self.assertEqual(main.ast_generate("a"), Chr('a'))

            def test_double_underscore_raises_syntax_error(self):
                self.assert_raises_syntax_error("a_a_a")

            def assert_raises_syntax_error(self, a):
                with self.assertRaises(ValueError) as cm:
                    main.ast_generate(a)
                self.assertEqual(str(cm.exception), "Syntax error in input!")

            def test_double_superscript_raises_syntax_error(self):
                self.assert_raises_syntax_error("a^a^a")

            def test_superscript(self):
                ast = SuperSub(Chr('a'), Chr('b'), LambdaExpr())
                superScriptStr = "a^b"
                self.assert_equal_ast(ast, superScriptStr)

            def assert_equal_ast(self, ast, superScriptStr):
                parsedAst = main.ast_generate(superScriptStr)
                self.assertEqual(parsedAst, ast)

            def test_subscript(self):
                ast = SubSuper(Chr('a'), Chr('b'), LambdaExpr())
                self.assert_equal_ast(ast, "a_b")

            def test_subscript_and_superscript(self):
                ast = SubSuper(Chr('a'), Chr('b'), SuperSuffix(Chr('c')))
                self.assert_equal_ast(ast, "a_b^c")

            def test_superscript_and_subscript(self):
                ast = SuperSub(Chr('a'), Chr('b'), SubSuffix(Chr('c')))
                self.assert_equal_ast(ast, "a^b_c")

            def test_qonda(self):
                #Esto deberia fallar no?
                self.assert_raises_syntax_error("a^b_c^d")
                self.assert_raises_syntax_error("a_b^c_d")

            def test_div(self):
                ast = DivExpr(Chr('a'), Chr('b'))
                self.assert_equal_ast(ast, "a/b")

            def test_div_minus(self):
                ast = DivExpr(Chr('a'), Concat(Concat(Chr('b'), Chr('-')),Chr('c')))
                self.assert_equal_ast(ast, "a/b-c")

            def test_parse_complex_formula(self):
                leftDivAst = Concat(SuperSub(Chr('A'), Chr('B'), LambdaExpr()), SuperSub(Chr('C'), Chr('D'), LambdaExpr()))
                rightDivAst = Concat(Concat(SuperSub(Chr('E'), Chr('F'), SubSuffix(Chr('G'))), Chr('+')), Chr('H'))
                parentesisedAst = GroupedPar(DivExpr(leftDivAst, rightDivAst))
                ast = Concat(Concat(parentesisedAst, Chr('-')), Chr('I'))

                self.assert_equal_ast(ast, "(A^BC^D/E^F_G+H)-I")

            def test_complex_formula_equal_formula_with_curly_brackets(self):
                astComplexFormula = main.ast_generate("(A^BC^D/E^F_G+H)-I")
                curlyBracketsFormula = "{({{A^B}{C^D}}/{{{E^F_G}+}H})-}I"
                astCurlyBrackets = main.ast_generate(curlyBracketsFormula)
                self.assertEqual(astComplexFormula, astCurlyBrackets)

        if __name__ == '__main__':
            unittest.main()

    \end{lstlisting}


    \subsection{main.py}

        \begin{lstlisting}[language=Python]
            #!/usr/bin/python3
            from sys import argv, exit
            import argparse

            from ply.lex import lex
            import tokrules

            from ply.yacc import yacc
            import parser_rules

            import AST_visitors

            def ast_generate(input_str):
                lexer = lex(module=tokrules)
                parser = yacc(module=parser_rules)
                ast = parser.parse(input_str, lexer)
                return ast

            def generar(input):
                ast = ast_generate(input)
                # print(ast)
                ast.accept(AST_visitors.EscaleVisitor(1))
                ast.accept(AST_visitors.WidthVisitor())
                ast.accept(AST_visitors.HVisitor())
                ast.accept(AST_visitors.XVisitor(0))
                ast.accept(AST_visitors.YVisitor(ast.h1))
                ast.accept(AST_visitors.SVGRendererVisitor())

                return ast

            if __name__ == "__main__":
                argparser = argparse.ArgumentParser(description='Parsea y genera SVG. Lo podes mandar a un archivo')
                argparser.add_argument('expression', type=str,
                                    help='Expresion para parsear')

                argparser.add_argument('--output', '-o', dest='output_filename', default='output.svg', type=str,
                                    help="Archivo output svg")

                argparser.add_argument('--print-svg', '-p', dest='print_svg', action='store_const',
                                const=True, default=False,
                                help='Imprimir svg')

                args = argparser.parse_args()

                ast_with_attributes = generar(args.expression)
                result = "<svg xmlns=\"http://www.w3.org/2000/svg\" width=\"{}\" height=\"{}\" version=\"1.1\" style=\"background: white\">\n<g transform=\"scale(40) translate(1,1)\" font-family=\"Courier\">\n".format(ast_with_attributes.a*50+80, (ast_with_attributes.h1+ast_with_attributes.h2)*40+80, ast_with_attributes.h1)
                result += ast_with_attributes.svg
                result += "</g>\n</svg>\n"

                output_file = open(args.output_filename, "w")
                output_file.write(result)
                output_file.close()

                if args.print_svg:
                    print(result)

        \end{lstlisting}
