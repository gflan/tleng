% ******************************************************** %
%              TEMPLATE DE INFORME ORGA2 v0.1              %
% ******************************************************** %
% ******************************************************** %
%                                                          %
% ALGUNOS PAQUETES REQUERIDOS (EN UBUNTU):                 %
% ========================================
%                                                          %
% texlive-latex-base                                       %
% texlive-latex-recommended                                %
% texlive-fonts-recommended                                %
% texlive-latex-extra?                                     %
% texlive-lang-spanish (en ubuntu 13.10)                   %
% ******************************************************** %


\documentclass[a4paper]{article}
\usepackage[spanish]{babel}
\usepackage[utf8]{inputenc}
\usepackage{charter}   % tipografia
\usepackage{graphicx}
\graphicspath{ {./imagenes/} }
\usepackage{tikz}
\usepackage{algpseudocode}
%\usepackage{makeidx}
\usepackage{paralist} %itemize inline


%\usepackage{float}
%\usepackage{amsmath, amsthm, amssymb}
%\usepackage{amsfonts}
%\usepackage{sectsty}
%\usepackage{charter}
%\usepackage{wrapfig}
%\usepackage{listings}
%\lstset{language=C}

% \setcounter{secnumdepth}{2}
\usepackage{underscore}
\usepackage{caratula}
\usepackage{url}
\usepackage{ragged2e}

\usepackage{titlesec}
\usepackage{parskip}

% ********************************************************* %
% ~~~~~~~~              Code snippets             ~~~~~~~~~ %
% ********************************************************* %

\usepackage{color} % para snipets de codigo coloreados
\usepackage{fancybox}  % para el sbox de los snipets de codigo

\newcommand{\quotes}[1]{``#1''}

\definecolor{litegrey}{gray}{0.94}

\newenvironment{codesnippet}{%
	\begin{Sbox}\begin{minipage}{\textwidth}\sffamily\small}%
	{\end{minipage}\end{Sbox}%
		\begin{center}%
		\vspace{-0.4cm}\colorbox{litegrey}{\TheSbox}\end{center}\vspace{0.3cm}}



% ********************************************************* %
% ~~~~~~~~         Formato de las páginas         ~~~~~~~~~ %
% ********************************************************* %

\usepackage{fancyhdr}
\pagestyle{fancy}

%\renewcommand{\chaptermark}[1]{\markboth{#1}{}}
\renewcommand{\sectionmark}[1]{\markright{\thesection\ - #1}}

\fancyhf{}

\fancyhead[LO]{Sección \rightmark} % \thesection\
\fancyfoot[LO]{\small{Bokser Brian, Lancioni Franco, Scherman Jonathan}}
\fancyfoot[RO]{\thepage}
\renewcommand{\headrulewidth}{0.5pt}
\renewcommand{\footrulewidth}{0.5pt}
\setlength{\hoffset}{-0.8in}
\setlength{\textwidth}{16cm}
%\setlength{\hoffset}{-1.1cm}
%\setlength{\textwidth}{16cm}
\setlength{\headsep}{0.5cm}
\setlength{\textheight}{25cm}
\setlength{\voffset}{-0.7in}
\setlength{\headwidth}{\textwidth}
\setlength{\headheight}{13.1pt}

\renewcommand{\baselinestretch}{1.1}  % line spacing

% ******************************************************** %

 \usepackage{setspace}

\begin{document}
\spacing{1.5}
\thispagestyle{empty}
\materia{Teoría de Lenguajes}
\submateria{Segundo Cuatrimestre de 2017}
\titulo{Trabajo Práctico}
\subtitulo{}
\integrante{Bokser Brian}{155/15}{brian.bokser@gmail.com}
\integrante{Franco Lancioni}{234/15}{gianflancioni@gmail.com}
\integrante{Jonathan Scherman}{152/15}{jonischerman@gmail.com}

\maketitle
\newpage
\tableofcontents
\newpage

\titlespacing\section{0pt}{8pt plus 4pt minus 2pt}{0pt plus 2pt minus 2pt}
\titlespacing\subsection{0pt}{12pt plus 4pt minus 2pt}{0pt plus 2pt minus 2pt}
\titlespacing\subsubsection{0pt}{8pt plus 4pt minus 2pt}{0pt plus 2pt minus 2pt}


\newcommand\produces{$\rightarrow$ \qquad}
\newcommand\alsoproduces{$\vert$}
\newcommand\super{$\string^$}
\newcommand\sub{$\_$}

%\normalsize
\newpage

\section{Introducción}
En este trabajo práctico desarrollamos un compositor de fórmulas
matemáticas. El mismo toma como entrada la descripción de una fórmula
en una versión simplificada del lenguaje utilizado por $LATEX$ y produce como salida un archivo SVG (Scalable Vector Graphics).
Esta versión simplificada nos permite usar utilizar la división (o fracciones) y utilizar subíndices o superíndices.

Usamos la librería PLY \cite{ply} de Python que nos permite hacer tanto el análisis léxico, que consiste en identificar subexpresiones de una cadena texto con símbolos terminales de nuestra gramática, como sintáctico, que consiste en trabajar sobre la estructura relativa a dicha gramática de la cadena input. Modificamos la gramática dada por enunciado, agregando además precedencias para que sea LALR.

Hacia el final, hicimos pruebas de los resultados generados por nuestro programa frente a fórmulas variadas, con muy buenos resultados.

\newpage

\section{Desarrollo}

\subsection{Lexer}

Utilizamos los siguientes tokens con sus respectivas reglas
\begin{itemize}
	\item \textbf{DIVIDE}: El símbolo '/' para la división
	\item Los literales: \quotes{\sub \ \super \ () \{\}}
	\item \textbf{CHR}: Cualquier símbolo exceptuando los anteriores.
	\\Esto incluye:
	\begin{itemize}
		\item cualquier caracter a-z y A-Z
		\item +
		\item *
	\end{itemize}
\end{itemize}

\subsection{Gramática}

$ \mathcal{G} = \langle$ \{$E$, $UNARYEXP$, $SUPEREXP$, $SUBEXP$\},   \big \{$CHR$, $DIVIDE$, `_', `\super', `(', `)', `\{', `\}' \big \},   $\mathcal{P}$,   $E$ $\rangle $

\begin{figure}[h!] \centering
\begin{tabular}{lrrl}
$\mathcal{P}:$
& $E$  & \produces     & $UNARYEXP$ \\
& & \alsoproduces & $E \verb| | E$ \\
& & \alsoproduces & $E$ \textbf{DIVIDE} $E$ \\
& & \alsoproduces & $UNARYEXP \verb| ^ | UNARYEXP \verb| | SUBEXP$ \\
& & \alsoproduces & $UNARYEXP \verb| _ | UNARYEXP \verb| | SUPEREXP$ \\
& & \alsoproduces & $E \verb| DIVIDE | E$ \\
& $UNARYEXP$  & \produces     & \textbf{CHR} \\
& & \alsoproduces & ( $E$ ) \\
& & \alsoproduces & \{ $E$ \} \\
& $SUPEREXP$  & \produces     &  \verb| ^ | $UNARYEXP$ \\
& & \alsoproduces & $\lambda$ \\
& $SUBEXP$  & \produces     &  \verb| _ | $UNARYEXP$ \\
& & \alsoproduces & $\lambda$ \\


\end{tabular}
\caption{Producciones de la gramática}
\label{fig:gramatica}
\end{figure}

Siendo la gramática original la siguiente:

\begin{figure} [ht]
    \centering
    \includegraphics[width=0.9\textwidth, height=0.22\textheight, keepaspectratio]{gramoriginal}
    \caption{Gramática original a parsear según enunciado.}
    \label{fig:gram_original}
\end{figure}

Como \emph{PLY}, la herramienta que usamos tanto para el análisis lexicográfico (lexer) como para parsear, usa técnicas de tablas LALR, tuvimos que hacer algunas modificaciones sobre la gramática original para poder generar una tabla de dichas características. \newline

Una de ellas fue juntar las producciones \emph{"superíndice"} y \emph{"super y subíndice"} en una única usando un no-terminal nuevo $SUBEXP$ que pudiera ser anulable o generar el subíndice e idem para las producciones \emph{"subíndice"} y \emph{"sub y superíndice"}. De lo contrario una cadena con super y subíndices  como $A\super B\sub C$ podría ser generada produciendo primero un superíndice y luego un subíndice usando dos producciones tanto como usando una única producción. \newline

Otra manera de sortear este problema hubiera sido eliminando las producciones que combinan super y subíndices y usando ambas producciones de manera consecutiva, pero preferimos mantener las producciones ternarias sobre las binarias para facilitar el recorrido y "decorado" de la estructura sintáctica (sino no se nos complicaría distinguir estructuras de cadenas como $ A\sub \{B\super C\} $ ó $\{A\sub B\}\super C$ de las de $A\sub B\super C$). \newline

El otro cambio importante fue agregar otro no-terminal $UNARYEXP$ que genera producciones 'unarias' (paréntesis o llaves sobre una $E$, o un $CHR$). La idea viene de la necesidad de desambiguar expresiones como $E\super E\sub E$ que podrían ser generadas como $E \Rightarrow E \super  E\ SUBEXP \Rightarrow E\super E\sub E$ o como $E \Rightarrow  E \super  E\ SUBEXP \Rightarrow E \super  E \Rightarrow E\ \super  E\sub \ E\ SUPEREXP \Rightarrow E \super  E\sub \ E\ $. \newline

Como según el enunciado ni los '\super ' ni los '_' son asociativos y además $E\sub  E \super  E$ y $E\super  E \sub  E$ son equivalentes, viendo expresiones de la pinta $E_1 \sub  E_2 \super  E_3$ se puede ver que ninguno de los tres no-terminales pueden producir nunca otro subíndice o superíndice en el mismo 'scope' de paréntesis o llaves. Esto es porque siempre podríamos invertir el orden usando la equivalencia mencionada anteriormente de modo que se asocien dos de estos símbolos. \\
Tampoco pueden producir concatenaciones o divisiones porque '\super' y '_' tienen mayor precedencia que estos, por ejemplo la cadena $A\super BC$ (escribiendo los terminales $CHAR$ como sus valores para mayor claridad) no tiene la estructura de $A\super \{BC\}$ sino de $A\super B$ concatenado a $C$ \footnote{idem para subindexación} y en el caso de $A/B\super C\sub D$ primero se resuelve la indexación de B y luego la división por lo tanto la estructura se corresponde a $A/\{B\super C\sub D\}$ \footnote {idem si fuera concatenación}.

Con todos estos cambios aún seguimos teniendo problemas de tipo \emph{Shift/Reduce} y \emph{Reduce/Reduce} en nuestras tablas, que resolvimos declarando las siguientes precedencias:

Tabla de precedencia (en orden creciente)
\begin{itemize}
	\item \textbf{DIV} asociativa a izquierda

    \item '\{', '(' asociativas a izquierda
    \item \textbf{CHR} asociativa a izquierda
    \item CONCAT, asociativa a izquierda, pseudosímbolo para la concatenación
    \item \verb|'^'| no asociativa
    \item \verb|'_'| no asociativa
\end{itemize}

Las reglas de la concatenación, división e indexación siguen la descripción del enunciado mientras que las que se corresponden a $Primeros(E)$ sirven para resolver en favor de \emph{Shifts} cuando se llega al final de la expresión de una división y se está por ver una concatenación. (es decir, \textbf{DIV} tiene menos precedencia que los 'primeros' de $E$ y que las otras operaciones) y que se tome \emph{Reduce} cada vez que se vió una concatenación si se está por ver una expresión concatenada o de división \footnote{Esto en teoría alcanzaría con declarar a CONCAT como asociativa a izquierda y de mayor precedencia que la división, pero al parecer para que \emph{Yacc} pueda resolver conflictos comparando orden de precedencias de símbolos con órdenes de precedencias de producciones hace falta cierta completitud sobre las declaraciones de precedencia de los símbolos que puedan llegar a ser el token corriente al decidir si reducir o no usando una producción.}.

\subsection{AST}

Como \emph{Yacc} solo nos permite sintetizar atributos únicamente en una \quotes{pasada} sobre el árbol de parsing y como veremos en la sección de atributos requerimos requerimos de varias pasadas para setear los atributos deseados, decidimos sintetizar como atributo de la gramática justamente su \emph{AST} para luego poder recorrerlo múltiples veces decorándolo. \newline

La idea es sencilla: por cada producción de la gramática sintetizamos un nodo que simboliza una operación sobre sus subtérminos. Por ejemplo, el input \texttt{A/B+C\super D\sub E} se corresponde con la estructura: \newline \texttt{DivExpr(Chr(A),Concat(Concat(Chr(B),Chr(+)),SuperSub(Chr(C), Chr(D), SubSuffix(Chr(E)))))
}

\subsection{Atributos y SVG}

Para poder generar un SVG partiendo del AST ya sintetizado, necesitamos decorarlo con atributos:

\begin{itemize}
	\item \textbf{e}: el interlineado o escala \emph{(heredado)}
	\item \textbf{a}: ancho de una expresión \emph{(sintetizado)}
	\item \textbf{h1}: corrimiento \quotes{para arriba} desde la base de una expresión \emph{(sintetizado)}
	\item \textbf{h2}: corrimiento \quotes{para abajo} desde la basa de una expresión \emph{(sintetizado)}
	\item \textbf{x}: ubicación en el eje x de una expresión \emph{(heredado)}
	\item \textbf{y}: ubicación en el eje y (la esquina \texttt{(0,0)} se corresponde con la esquina superior izquierda del buffer) de una expresión \emph{(heredado)}
	\item \textbf{svg}: output en términos de tags SVG generados por la expresión \emph{(sintetizado)}
\end{itemize}

La manera de hacer pasadas sobre el AST se corresponde a la metodología del patrón de diseño \emph{\quote{Visitors}} comunmente usado para iterar este tipo de estructuras y que permite usar polimorfismo y double-dispatch para hacer llamados recursivos sobre cada nodo.

\subsubsection{Cálculo de atributos}

\begin{itemize}
	\item El atributo \textbf{\quotes{e}} correspondiente a la escala, se inicia en 1 sobre la raíz y se reduce a un 70\% de su valor en cada indexación. Se trata de un atributo heredado.

	\item El atributo \textbf{\quotes{a}} correspondiente al ancho es:
	\begin{itemize}
		\item el 60\% de la escala para caracteres
		\item suma de anchos de subexpresiones el caso de concatenaciones
		\item suma del ancho de la expresión principal y máximo ancho entre los índices para indexación
		\item máximo entre ancho del numerador y del denominador para divisiones
		\item suma de la expresión mas ancho de los dos paréntesis/llaves en dichos casos
	\end{itemize}

	\item El atributo \textbf{\quotes{h1}} es:
	\begin{itemize}
		\item el interlineado para caracteres
		\item máximo entre ambos h1 para concatenaciones
		\item máximo entre el h1 de la expresión principal y h1 del superíndice mas su altura desde la base de la expresión general
		\item h1 del numerador menos el corrimiento para arriba desde la base de la línea de división para tales expresiones
		\item h1 de la expresión principal para paréntesis/llaves
	\end{itemize}

	\item El atributo \textbf{\quotes{svg}} se toma concatenando los strings de las subexpresiones (y agregando líneas, caracteres y paréntesis necesarios con los atributos heredados)

\end{itemize}

\clearpage

\clearpage

\end{document}
